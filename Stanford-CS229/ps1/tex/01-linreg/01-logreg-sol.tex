\begin{answer}
    \begin{equation}
        \frac{\partial}{\partial \theta_{j}}J(\theta) = -\frac{1}{m}\sum_{i=1}^{m}(y^{(i)}-h_{\theta}(x^{(i)}))x_j^{(i)} \\
    \end{equation}
    So
    \begin{align}
        \frac{\partial J(\theta)}{\partial \theta_{j}\partial \theta_{k}} = \frac{1}{m}\sum_{i=1}^{m}(h_{\theta}(x^{(i)}))(1-h_{\theta}(x^{(i)}))x_j^{(i)}x_k^{(i)}
    \end{align}
    Therefore for any vector z:
    \begin{align}
        z^T H z &= \frac{1}{m}\sum_{i=1}^{m}\sum_{j,k=1}^{m}(h_{\theta}(x^{(i)}))(1-h_{\theta}(x^{(i)}))x_j^{(i)}x_k^{(i)}z_j z_k\\
                &= \frac{1}{m}\sum_{i=1}^{m}\sum_{j,k=1}^{m}(h_{\theta}(x^{(i)}))(1-h_{\theta})((x^{(i)})^Tz)^2 
    \end{align}
    And because $h_0$ uses the sigmoid function, which outputs numbers between 0 and 1, and $((x^{(i)})^Tz)^2$ is a square meaning it can only be positive.\\
    Hence for any vector z: $z^THz \geq 0$
\end{answer}